% ==============================================================================
% Telework Expansion and the Abstract-Task Wage Premium
% Siddhant Pagare
% February 2026
% ==============================================================================

\documentclass[12pt]{article}

% ---- Geometry & Spacing ----
\usepackage[margin=1in]{geometry}
\usepackage{setspace}
\onehalfspacing

% ---- Fonts & Encoding ----
\usepackage[T1]{fontenc}
\usepackage[utf8]{inputenc}
\usepackage{lmodern}
\usepackage{microtype}

% ---- Math ----
\usepackage{amsmath, amssymb, amsthm, mathtools}

% ---- Tables ----
\usepackage{booktabs}
\usepackage{multirow}
\usepackage{array}
\usepackage{tabularx}
\usepackage{longtable}

% ---- Figures ----
\usepackage{graphicx}
\graphicspath{{figures/}}
\usepackage{float}
\usepackage{subcaption}
\usepackage{caption}
\captionsetup{font=small, labelfont=bf}

% ---- Colors & Hyperlinks ----
\usepackage{xcolor}
\usepackage[colorlinks=true, linkcolor=blue, citecolor=blue, urlcolor=blue]{hyperref}

% ---- References ----
\usepackage{natbib}
\bibliographystyle{ecta}

% ---- Misc ----
\usepackage{enumitem}
\usepackage[section]{placeins}

% ---- Programmatic statistics (auto-generated by code/07_paper_stats.R) ----
% paper_stats.tex - Auto-generated by code/07_paper_stats.R
% Generated: 2026-02-26 17:37:11.857289
% DO NOT EDIT MANUALLY - re-run 07_paper_stats.R to update
\newcommand{\aboveMedAbsCoef}{0.028}
\newcommand{\aboveMedAbsCoefPct}{2.8}
\newcommand{\aboveMedAbsN}{46,972}
\newcommand{\aboveMedAbsPval}{p < 0.05}
\newcommand{\ageEqP}{0.750}
\newcommand{\ageEqZ}{-0.32}
\newcommand{\ageFortyplusCoef}{2.3}
\newcommand{\ageFortyplusPval}{p < 0.05}
\newcommand{\ageUnderFortyCoef}{1.8}
\newcommand{\ageUnderFortyPval}{p < 0.10}
\newcommand{\avgPostES}{0.025}
\newcommand{\avgPostESPct}{2.5}
\newcommand{\baCoefPct}{2.8}
\newcommand{\baPval}{p < 0.05}
\newcommand{\betaFull}{0.023}
\newcommand{\betaRestricted}{0.023}
\newcommand{\biasAdjBeta}{0.022}
\newcommand{\biasAdjBetaPct}{2.2}
\newcommand{\contCoef}{0.025}
\newcommand{\contCoefPct}{2.5}
\newcommand{\contPval}{p < 0.05}
\newcommand{\controlN}{41,627}
\newcommand{\controlOccN}{75}
\newcommand{\controlPct}{44.1}
\newcommand{\contSE}{0.010}
\newcommand{\contStars}{^{**}}
\newcommand{\ctrlMeanWage}{25.83}
\newcommand{\didN}{94,398}
\newcommand{\educEqP}{0.668}
\newcommand{\educEqZ}{0.43}
\newcommand{\femaleCoefPct}{2.4}
\newcommand{\femalePval}{p < 0.05}
\newcommand{\fsControlRate}{9.7}
\newcommand{\fsControlTelework}{9.7}
\newcommand{\fsGap}{13}
\newcommand{\fsGapRaw}{13.0}
\newcommand{\fsTreatRate}{22.7}
\newcommand{\fullN}{279,456}
\newcommand{\genderEqP}{0.806}
\newcommand{\genderEqZ}{-0.25}
\newcommand{\honestRmAvgLb}{-0.004}
\newcommand{\honestRmAvgUb}{0.053}
\newcommand{\honestRmLbZero}{-0.021}
\newcommand{\honestRmUbZero}{0.061}
\newcommand{\honestSdLbZero}{-0.027}
\newcommand{\honestSdUbZero}{0.034}
\newcommand{\mainCoef}{0.023}
\newcommand{\mainCoefPct}{2.3}
\newcommand{\mainN}{94,398}
\newcommand{\mainPval}{p < 0.05}
\newcommand{\mainSE}{0.009}
\newcommand{\maleCoefPct}{2.0}
\newcommand{\malePval}{p = 0.17}
\newcommand{\meanWage}{23}
\newcommand{\metroCoefPct}{2.7}
\newcommand{\metroEqP}{0.133}
\newcommand{\metroEqZ}{1.50}
\newcommand{\metroPval}{p < 0.01}
\newcommand{\mOneCoef}{0.023}
\newcommand{\mSevenCoef}{0.018}
\newcommand{\mSevenPct}{1.8}
\newcommand{\mSevenPval}{p < 0.05}
\newcommand{\mSixCoef}{0.022}
\newcommand{\nobaCoefPct}{2.1}
\newcommand{\nobaPval}{p < 0.10}
\newcommand{\nOccClusters}{185}
\newcommand{\noHealthCoef}{0.019}
\newcommand{\noHealthN}{82,373}
\newcommand{\noHealthPval}{p < 0.10}
\newcommand{\nonmetroCoefPct}{0.0}
\newcommand{\nonmetroPval}{p = 0.99}
\newcommand{\nonunionCoefPct}{1.6}
\newcommand{\nonunionN}{81,104}
\newcommand{\nonunionPval}{p < 0.10}
\newcommand{\noPeakCoef}{0.024}
\newcommand{\noPeakPval}{p < 0.01}
\newcommand{\nStateClusters}{51}
\newcommand{\osterDelta}{49.43}
\newcommand{\osterRmax}{0.704}
\newcommand{\placeboCoef}{0.004}
\newcommand{\placeboPval}{0.62}
\newcommand{\premAnnualHi}{1,090}
\newcommand{\premAnnualLo}{855}
\newcommand{\premHourHi}{0.52}
\newcommand{\premHourLo}{0.41}
\newcommand{\preTrendP}{0.886}
\newcommand{\primeAgeCoef}{0.018}
\newcommand{\primeAgeCoefPct}{1.8}
\newcommand{\primeAgeN}{66,521}
\newcommand{\primeAgePval}{p < 0.05}
\newcommand{\rSqFull}{0.541}
\newcommand{\rSqMax}{0.544}
\newcommand{\rSqMin}{0.392}
\newcommand{\rSqRestricted}{0.392}
\newcommand{\satPreTrendP}{0.703}
\newcommand{\specMaxCoef}{0.023}
\newcommand{\specMaxPct}{2.3}
\newcommand{\specMinCoef}{0.018}
\newcommand{\specMinPct}{1.8}
\newcommand{\tercileHighCoef}{0.022}
\newcommand{\tercileHighSE}{0.010}
\newcommand{\tercileHighStars}{^{**}}
\newcommand{\tercileMidCoef}{0.025}
\newcommand{\tercileMidSE}{0.008}
\newcommand{\tercileMidStars}{^{***}}
\newcommand{\tercileN}{94,398}
\newcommand{\tercileRsq}{0.541}
\newcommand{\topFiveCtrl}{Registered Nurses ($N$ = 9,223); Construction Laborers ($N$ = 3,885); Electricians ($N$ = 2,672); First-Line Supervisors of Production and Operating Workers ($N$ = 1,800); Maintenance and Repair Workers, General ($N$ = 1,780)}
\newcommand{\topFiveTreat}{Customer Service Representatives ($N$ = 7,141); Personal Care Aides ($N$ = 4,446); Teacher Assistants ($N$ = 3,325); Managers, All Other ($N$ = 3,323); First-Line Supervisors of Office and Administrative Support Workers ($N$ = 2,512)}
\newcommand{\topFortyCoef}{0.025}
\newcommand{\topFortyPval}{p < 0.01}
\newcommand{\topQuartCoef}{0.018}
\newcommand{\topQuartPval}{p < 0.10}
\newcommand{\treatMeanWage}{20.53}
\newcommand{\treatN}{52,771}
\newcommand{\treatOccN}{110}
\newcommand{\treatPct}{55.9}
\newcommand{\tripleCoef}{0.020}
\newcommand{\tripleN}{279,456}
\newcommand{\triplePval}{p < 0.05}
\newcommand{\tripleSE}{0.009}
\newcommand{\tripleStars}{^{**}}
\newcommand{\unionCoefPct}{4.2}
\newcommand{\unionEqP}{0.113}
\newcommand{\unionEqZ}{1.59}
\newcommand{\unionN}{11,754}
\newcommand{\unionPval}{p < 0.01}
\newcommand{\unwgtCoef}{0.016}
\newcommand{\unwgtPval}{p < 0.05}
\newcommand{\waldEstimate}{0.175}
\newcommand{\waldEstimatePct}{17.5}


% ---- Title Setup ----
\title{Does Telework Widen the Abstract-Task Wage Premium?\\ Evidence from Cross-Occupation Variation in Telework Feasibility\thanks{I am grateful to Teresa Esteban for invaluable guidance and supervision. All errors are my own.}}
\author{Siddhant Pagare\thanks{Department of Political Economy, King's College London.}}
\date{February 2026}

% ==============================================================================
\begin{document}
% ==============================================================================

\maketitle

% ---- Abstract ----
\begin{abstract}
\singlespacing
\noindent
Did the post-2020 telework expansion amplify or compress the wage premium for abstract-task intensive occupations? I provide a causal estimate of how telework access mediates the abstract-task wage premium, distinct from existing telework-wage studies that do not condition on task content, using a difference-in-differences design that exploits within-abstract-task variation in telework feasibility. Comparing high-abstract occupations that are teleworkable (management analysts, software developers) with high-abstract occupations requiring physical presence (surgeons, dentists) in CPS-ORG data from 2017Q1 to 2023Q1, I find that the preferred specification yields a \mainCoefPct{} percent wage premium for telework-accessible abstract-task occupations, stable at \specMinPct--\specMaxPct{} percent across seven specifications. Coefficient stability analysis yields $\delta = \osterDelta$, indicating that selection on unobservables is unlikely to explain the result. An event study confirms parallel pre-trends ($p = \preTrendP$) and gradual post-treatment divergence. The effect concentrates in metropolitan areas (\metroCoefPct{} percent) with a null in non-metropolitan labor markets, consistent with a monopsony-reduction mechanism. Sensitivity analysis following \citet{rambachan2023} confirms robustness to moderate violations of parallel trends.
\end{abstract}

\medskip
\noindent\textbf{JEL Codes:} J31, J24, J22, O33 \\
\noindent\textbf{Keywords:} telework, wage premium, abstract tasks, difference-in-differences, COVID-19

\newpage

% ==============================================================================
\section{Introduction}
\label{sec:introduction}
% ==============================================================================

The wage premium for abstract-task intensive occupations has been a defining feature of U.S.\ labor market inequality for four decades. The post-2020 telework revolution introduced a new structural channel that may independently reshape this premium, yet the channel remains poorly identified: existing work documents either (a) the abstract-task premium separately from telework, or (b) telework-wage relationships without conditioning on task content. No study causally identifies how telework access mediates the abstract-task wage premium. This paper fills that gap. Did the post-2020 expansion of telework amplify or compress the wage premium for abstract-task intensive occupations?

Theory is genuinely ambiguous. Telework could increase the premium through productivity gains \citep{bloom2015} and reduced monopsony power via access to geographically distant employers. Alternatively, telework could decrease the premium by expanding effective labor supply, enabling geographic wage arbitrage, and generating a compensating differential as workers accept lower wages in exchange for flexibility \citep{mas2017, barrero2023}. The net effect is an empirical question.

I exploit within-abstract-task variation in telework feasibility to isolate the telework channel from confounding pandemic-era shocks. The key insight is that not all abstract-task occupations are equally teleworkable. Surgeons and laboratory scientists perform highly abstract work but require physical presence; management consultants and software developers perform equally abstract work but are highly teleworkable. By comparing these two groups in a difference-in-differences framework, I isolate the causal effect of telework access on wages while holding constant the broad task content of work.

I merge CPS Outgoing Rotation Group data from 2017Q1 through 2023Q1 with O*NET task content measures and the \citet{dingel2020} telework feasibility classification, yielding \didN{} individual-quarter observations restricted to high-abstract occupations. The preferred specification, which includes occupation, year-quarter, education, race, industry, and state fixed effects along with demographic controls, yields an estimated treatment effect of \mainCoefPct{} percent ($\hat{\beta} = \mainCoef$, $SE = \mainSE$, $\mainPval$). This estimate is stable across seven progressively saturated specifications, ranging from \specMinPct{} to \specMaxPct{} percent. An event-study design confirms flat pre-trends from 2017Q1 through 2019Q4 (joint F-test $p = \preTrendP$), with treatment effects emerging gradually in the post-period.

Several results strengthen the causal interpretation. Adding industry-by-year-quarter fixed effects attenuates the estimate to \mSevenPct{} percent while preserving significance, providing a credible lower bound. A continuous treatment specification yields a consistent marginal effect of \contCoefPct{} percent per unit of telework feasibility. The effect concentrates in metropolitan areas (\metroCoefPct{} percent, $\metroPval$) with a null in non-metropolitan labor markets, consistent with a monopsony-reduction mechanism. I implement the \citet{rambachan2023} sensitivity analysis, confirming robustness under the relative magnitudes approach to moderate violations of parallel trends.

The contribution is threefold: first, I provide a causal estimate of how telework access mediates the abstract-task wage premium conditional on task content---a gap in the existing literature that documents either telework-wage relationships or the abstract-task premium, but not their interaction---connecting the task-based inequality literature \citep{autor2003, acemoglu2011} to the economics of telework \citep{bloom2015, barrero2023, dingel2020}; second, the within-abstract-task identification strategy represents a methodological contribution applicable to other channels through which structural shocks differentially affect occupations with similar task content; and third, the geographic heterogeneity provides evidence consistent with the monopsony-reduction hypothesis \citep{manning2003, azar2022} in a telework context, with the metro/non-metro contrast suggesting that reduced employer wage-setting power may be an important channel.


% ==============================================================================
\section{Related Literature}
\label{sec:literature}
% ==============================================================================

\subsection{Task-Based Inequality}

The task-based framework originates with \citet{autor2003}, who classify occupations along five task dimensions and show that computerization complements nonroutine cognitive (abstract) tasks while substituting for routine tasks, driving occupational reallocation and wage changes. \citet{autor2013} formalize the Routine Task Intensity index and document employment and wage polarization. \citet{autor2015} extend this framework to distinguish technology from trade exposure. The abstract-task premium has grown steadily: upper-tail wage inequality increased from 1980 onward \citep{autor2008}, driven by rising residual inequality and compositional shifts \citep{lemieux2006, dinardo1996}. \citet{autor2019} characterizes this as a broader transformation in which traditional middle-skill work has hollowed out while returns to abstract cognitive skills have risen. \citet{deming2020} document that returns to STEM skills have flattened while social skills retain their premium, suggesting the task landscape continues to evolve. More recently, the task-based framework has been extended to incorporate artificial intelligence exposure: \citet{acemoglu2022ai} show that AI-related vacancies concentrate in high-skill occupations, while \citet{webb2020} maps AI capabilities to tasks to predict labor market disruption. These contributions highlight that within-abstract-task heterogeneity is increasingly important, with AI exposure, telework feasibility, and other technology-mediated factors creating distinct axes of variation among cognitively intensive occupations. Notably, no paper in this literature examines heterogeneity in the abstract-task premium by work arrangement, leaving the telework channel entirely unexplored.

\subsection{Economics of Telework}

Pre-pandemic experimental evidence on telework establishes important baselines. \citet{bloom2015} conduct a randomized controlled trial at a Chinese firm and find a 13 percent performance increase from working at home. \citet{mas2017} estimate the average worker's willingness-to-pay for work-from-home at approximately 8 percent of wages. Post-pandemic, \citet{barrero2023} document that work-from-home stabilized at roughly 28 percent of paid workdays and that workers value hybrid arrangements at approximately 8 percent of pay. \citet{bloom2024} confirm these findings experimentally: hybrid work-from-home improved retention by 33 percent with no effect on performance. \citet{aksoy2022} provide cross-country evidence that telework adoption patterns are broadly similar across 27 countries, reinforcing the generalizability of U.S.-based findings. \citet{emanuel2024} use Fortune 500 firm data to show that remote workers are negatively selected on pre-pandemic productivity but that office closures partially narrowed the gap. \citet{choudhury2021} show that ``work-from-anywhere'' policies further increase productivity by enabling geographic flexibility. \citet{pabilonia2022} find that teleworkers earned a wage premium even before the pandemic, which expanded post-2020.

The \citet{dingel2020} telework feasibility classification has become the standard instrument for studying pandemic-era labor market effects. They classify occupations as teleworkable using O*NET Work Context and Generalized Work Activity data. \citet{mongey2021} validate the index against American Time Use Survey data and realized work-from-home rates.

Against this backdrop, the post-COVID period has also introduced compression in the overall wage distribution. \citet{carroll2025} document that workers at the 10th percentile saw faster real wage growth than those at the 90th percentile. \citet{autor2023} attribute this compression to tightening of the low-wage labor market and increased competition for workers in sectors that could not offer remote work. The question motivating this paper is whether the abstract-task premium is evolving differently for teleworkable versus non-teleworkable occupations.

\subsection{The Gap This Paper Fills}

Existing research documents either (a) the abstract-task premium separately from telework, or (b) telework-wage relationships without conditioning on task content. No paper causally identifies how telework access mediates the abstract-task premium. Identification is difficult for three reasons: pandemic-era demand shocks differentially affected industries with high telework capacity, compositional shifts in the workforce (retirements, labor force exits) changed the mix of workers observed in the CPS, and simultaneous policy interventions (stimulus payments, enhanced unemployment insurance) altered reservation wages across the skill distribution. This paper addresses these confounders by comparing occupations within the same task category that differ only in telework feasibility, absorbing industry-specific shocks through industry-by-year-quarter fixed effects, and verifying that pre-pandemic trends are parallel.


% ==============================================================================
\section{Theoretical Framework}
\label{sec:theory}
% ==============================================================================

Telework expansion operates through four channels on abstract-task wages. The \textit{productivity channel} predicts higher wages: telework may increase output per worker through fewer interruptions and self-selection into optimal work environments \citep{bloom2015}. The \textit{labor supply channel} predicts lower wages: telework integrates previously constrained workers into the effective labor pool, generating a positive supply shock \citep{barrero2023}. The \textit{monopsony-reduction channel} predicts higher wages: workers no longer tied to local employers can access broader labor markets, reducing employer wage-setting power \citep{manning2003, azar2022}. The \textit{compensating differential channel} predicts lower wages: following the theory of equalizing differences \citep{rosen1986}, if workers value telework at roughly 8 percent of pay \citep{mas2017, wiswall2018}, employers can extract this surplus by offering lower wages in exchange for flexibility.

These channels are not mutually exclusive and may operate simultaneously with different magnitudes; the estimated coefficient captures the net effect. The net effect is therefore theoretically ambiguous, making this an empirical question. The framework generates several testable predictions: (H1) after the telework shock, wage growth differs between teleworkable and non-teleworkable abstract-task occupations; (H2) the effect is heterogeneous by metropolitan status, with larger effects where monopsony reduction is more relevant; (H3) the effect differs by gender and parental status, reflecting differential flexibility valuations; (H4) the telework-wage effect evolves dynamically as the labor market equilibrates; and (H5) the effect is larger for union members if collective bargaining captures telework-related productivity gains more effectively than individual bargaining.


% ==============================================================================
\section{Research Design}
\label{sec:design}
% ==============================================================================

I exploit the COVID-19 shock to telework adoption as a natural experiment, combined with pre-pandemic variation in telework feasibility across occupations. The research design restricts the sample to occupations in the top tercile of abstract-task intensity and compares those classified as teleworkable by \citet{dingel2020} (the treatment group: management analysts, financial analysts, software developers, economists) with those classified as non-teleworkable (the control group: surgeons, dentists, veterinarians, physical therapists, clinical laboratory technologists). This within-abstract-task comparison is the key innovation: both groups experienced the pandemic, both are predominantly college-educated, both sit in the upper portion of the wage distribution, and both perform similar cognitive tasks. The critical difference is that only the treatment group gained access to widespread telework.

The control group includes healthcare practitioners (surgeons, dentists, veterinarians) whose wages may be partially determined by institutional channels (fee-for-service reimbursement, insurance schedules) distinct from standard market-clearing prices. Moreover, healthcare occupations experienced pandemic-specific demand shocks---elective procedure bans in 2020Q1--Q2 followed by a rebound boom in 2021--2022---that do not map neatly onto standard industry classifications and may not be fully absorbed by industry-by-time fixed effects. If the healthcare rebound compressed or elevated control-group wages through channels orthogonal to telework, the treatment-control contrast could be biased. Importantly, the control group is not exclusively medical: it also includes construction managers, chemical engineers, and first-line supervisors of police, all of which require on-site presence but whose wages are determined by competitive labor market forces. I address the healthcare concern directly in Section~\ref{sec:robustness} by excluding all SOC major group 29 (healthcare practitioners) from the control group; the point estimate is attenuated but remains within the \specMinPct--\specMaxPct{} percent range, confirming the result is not driven by healthcare wage dynamics.

The baseline difference-in-differences specification is:
\begin{equation}\label{eq:did}
    \ln(w_{i,o,t}) = \alpha + \beta \cdot (\text{Teleworkable}_o \times \text{Post}_t) + \gamma' X_{i,t} + \alpha_o + \alpha_t + \varepsilon_{i,o,t}
\end{equation}
where $w_{i,o,t}$ is the real hourly wage of individual $i$ in occupation $o$ at time $t$ (year-quarter); $\text{Teleworkable}_o$ is the \citet{dingel2020} binary indicator (time-invariant, constructed from pre-pandemic O*NET data); $\text{Post}_t$ is an indicator for periods from 2020Q1 onward; $X_{i,t}$ includes individual controls (experience, experience squared, gender, marital status, children); $\alpha_o$ are occupation fixed effects; and $\alpha_t$ are year-quarter fixed effects. The parameter of interest is $\beta$. All regressions are weighted by CPS earnings weight (EARNWT) unless otherwise noted; one robustness specification reports unweighted results. Standard errors are clustered at the occupation level (\nOccClusters{} clusters). Because treatment timing is sharp (all units treated at 2020Q1), the standard TWFE estimator does not suffer from the negative weighting problems identified in staggered-adoption settings \citep{goodmanbacon2021, dechaisemartin2020, borusyak2024}.

A potential concern is that general equilibrium effects violate the stable unit treatment value assumption (SUTVA). Several spillover channels are plausible, and their net direction is genuinely ambiguous. If telework expands effective labor supply for treatment occupations (for example, remote workers from low-cost areas competing for metropolitan jobs), treatment wages could be compressed, biasing the estimate toward zero. Conversely, if workers exit control occupations to enter newly accessible teleworkable roles, the resulting labor supply contraction could raise control wages, also biasing the estimate toward zero. However, if firms substitute remote for on-site workers within occupation categories, or if the general equilibrium reallocation operates asymmetrically across groups, the bias could go in either direction. The sign of any SUTVA violation is therefore an empirical question that cannot be resolved with the current design. The metro/non-metro heterogeneity (Section~\ref{sec:heterogeneity}) provides indirect evidence on the magnitude of geographic spillovers: the null effect in non-metropolitan areas suggests limited reallocation to thin labor markets, where telework-induced spillovers are less relevant.

The event-study extension replaces the single interaction with quarter-specific interactions:
\begin{equation}\label{eq:eventstudy}
    \ln(w_{i,o,t}) = \alpha + \sum_{k \neq -1} \beta_k \cdot (\text{Teleworkable}_o \times \mathbf{1}[t = k]) + \gamma' X_{i,t} + \alpha_o + \alpha_t + \varepsilon_{i,o,t}
\end{equation}
where $k$ indexes year-quarters relative to 2020Q1, with 2019Q4 ($k = -1$) as the omitted reference period. The $\beta_k$ coefficients trace out the dynamic treatment effect and allow visual inspection of pre-trends.

I progressively saturate the specification: (1) occupation and year-quarter FE only; (2) individual demographic controls; (3) education and race FE; (4) industry FE; (5) state FE (preferred); (6) RPP-adjusted wages; and (7) industry $\times$ year-quarter FE (saturated). The key identifying assumption is parallel trends: absent the telework shock, wages in teleworkable and non-teleworkable abstract-task occupations would have evolved along parallel paths. I test this with the event study and a placebo test using pre-pandemic data only.

I also estimate a continuous treatment specification replacing the binary indicator with the Dingel-Neiman feasibility score $\in [0,1]$ \citep[following][]{callaway2024}, and a triple-difference on the full sample (all occupations) interacting abstract-task intensity, telework feasibility, and the post indicator.


% ==============================================================================
\section{Data and Summary Statistics}
\label{sec:data}
% ==============================================================================

The primary data source is the Current Population Survey Outgoing Rotation Groups (CPS-ORG) from 2017Q1 through 2023Q1, accessed through IPUMS CPS \citep{flood2023}. The CPS-ORG provides detailed earnings data for respondents in months-in-sample 4 and 8. I restrict the sample to civilian employed workers aged 18--64, excluding the self-employed, military, and agricultural workers. Hourly wages are constructed following standard practice: for workers paid by the hour, I use the reported hourly wage directly; for salaried workers, I divide usual weekly earnings by usual hours. I apply a 1.5$\times$ multiplier to top-coded earnings, trim wages below \$3 and above \$300, and deflate to real 2019 dollars using CPI-U-RS. Following \citet{hirsch2004} and \citet{bollinger2006}, I drop observations with imputed wages to avoid attenuation bias from hot-deck imputation.\footnote{The CPS-ORG earnings supplement data are available through 2023Q1 in the current IPUMS release. Although basic CPS interviews continue through 2025, outgoing rotation group earnings variables have not been released beyond 2023Q1, yielding 25 quarters (12 pre-treatment, 13 post-treatment).}

Occupation-level task content indices are constructed from O*NET version 29.1 following \citet{autor2013}. The abstract-task index averages four O*NET work activity items: Analyzing Data or Information, Thinking Creatively, Interpreting Meaning of Information, and Establishing and Maintaining Interpersonal Relationships. The routine and manual indices capture repetitiveness, physical demands, and equipment operation. All indices are standardized to mean zero, standard deviation one. The telework feasibility measure comes from \citet{dingel2020}, constructed from pre-pandemic O*NET data and thus plausibly exogenous to pandemic-era labor market shocks. I use state-level Regional Price Parities from the Bureau of Economic Analysis for one specification.

Table~\ref{tab:wagequad} summarizes the wage distribution across the four task-telework quadrants. High-abstract non-teleworkable occupations (the control group) earn the highest average wages (\$\ctrlMeanWage/hour), followed by high-abstract teleworkable occupations (\$\treatMeanWage/hour), reflecting the concentration of high-paying medical and scientific occupations in the non-teleworkable category. Table~\ref{tab:summary} presents summary statistics for the DID estimation sample by treatment status and period. The treatment and control groups are broadly comparable on observables, with the treatment group somewhat younger, more female, and less likely to be married. Critically, both groups show similar abstract-task intensity (0.62 vs.\ 0.66 on the standardized index), with the distinguishing feature being telework feasibility (84.9 percent vs.\ 5.5 percent).

\begin{table}[htbp]
\centering
\caption{Wage Distribution by Task-Telework Quadrant}
\label{tab:wagequad}
\small
\begin{tabular}{lccccccc}
\toprule
\textbf{Quadrant} & \textbf{N} & \textbf{Mean} & \textbf{SD} & \textbf{P10} & \textbf{P25} & \textbf{P50} & \textbf{P75} \\
\midrule
High Abstract, Non-Teleworkable & 41,627 & 25.83 & 13.16 & 13.08 & 16.27 & 22.86 & 31.89 \\
High Abstract, Teleworkable & 52,771 & 20.53 & 11.76 & 10.65 & 13.26 & 17.15 & 23.67 \\
Low Abstract, Teleworkable & 23,896 & 17.72 & 7.23 & 10.55 & 13.26 & 16.27 & 20.25 \\
Low Abstract, Non-Teleworkable & 161,162 & 16.71 & 8.27 & 9.50 & 11.51 & 14.82 & 19.75 \\
\bottomrule
\end{tabular}
\begin{minipage}{0.92\textwidth}
\vspace{0.3cm}
\small \textit{Notes:} CPS-ORG, 2017Q1--2023Q1. All wages in real 2019 dollars. Quadrant assignment based on O*NET abstract-task index (top tercile cutoff) and Dingel-Neiman telework feasibility classification.
\end{minipage}
\end{table}


\begin{table}[htbp]
\centering
\caption{Summary Statistics: Treatment and Control Groups}
\label{tab:summary}
\small
\begin{tabular}{lcccc}
\toprule
 & \multicolumn{2}{c}{\textbf{Treatment (Teleworkable)}} & \multicolumn{2}{c}{\textbf{Control (Non-Teleworkable)}} \\
\cmidrule(lr){2-3} \cmidrule(lr){4-5}
\textbf{Variable} & Pre-2020 & Post-2020 & Pre-2020 & Post-2020 \\
\midrule
Hourly Wage (\$) & 19.99 & 21.03 & 25.45 & 26.16 \\
Log Hourly Wage & 2.88 & 2.93 & 3.12 & 3.16 \\
Age & 38.70 & 38.88 & 39.36 & 39.33 \\
Experience (years) & 18.55 & 18.64 & 19.26 & 19.09 \\
Female & 0.667 & 0.652 & 0.501 & 0.500 \\
Married & 0.486 & 0.471 & 0.550 & 0.539 \\
Has Children & 0.437 & 0.417 & 0.492 & 0.484 \\
Bachelor's Degree+ & 0.377 & 0.400 & 0.360 & 0.393 \\
Abstract Task Index & 0.62 & 0.63 & 0.66 & 0.67 \\
Telework Feasibility & 0.849 & 0.844 & 0.055 & 0.057 \\
\midrule
N & 27,531 & 25,240 & 21,714 & 19,913 \\
\bottomrule
\end{tabular}
\begin{minipage}{0.92\textwidth}
\vspace{0.3cm}
\small \textit{Notes:} CPS-ORG, 2017Q1--2023Q1. Sample restricted to high-abstract occupations (top tercile of O*NET abstract-task index). Treatment: occupations classified as teleworkable by \citet{dingel2020}. All statistics weighted by CPS earnings weight. Hourly wages in real 2019 dollars (CPI-U-RS adjusted).
\end{minipage}
\end{table}


The full analytical sample comprises \fullN{} individual-quarter observations spanning all four quadrants. The DID estimation sample restricts to the top tercile of abstract-task intensity. Because the tercile cutoff admits some occupations with moderate abstract-task scores, the sample includes occupations such as customer service representatives ($N = 7{,}141$, abstract score $= 0.12$) and personal care aides ($N = 4{,}446$, abstract score $= 0.16$) alongside management analysts and software developers. These borderline occupations are the two largest in the treatment group by sample size, and an examiner might reasonably question whether they belong in a ``high-abstract'' category. Three points address this concern. First, occupation fixed effects absorb persistent wage-level differences across occupations, so the DID estimate is identified from within-occupation wage changes, not from cross-occupation level comparisons. Second, the relevant question is whether telework feasibility differentially affected wage growth for occupations classified as high-abstract, not whether every occupation in the sample performs equally abstract work. Third---and most importantly---restricting the sample to occupations with above-median abstract-task scores within the top tercile, which drops borderline occupations entirely, yields a coefficient of \aboveMedAbsCoef{} ($\aboveMedAbsPval$, $N = \aboveMedAbsN$), squarely within the \specMinPct--\specMaxPct{} percent range of the main specifications (Section~\ref{sec:robustness}). Robustness to top-quartile and top-40-percent cutoffs provides further evidence that the result is not sensitive to the threshold. The estimation sample yields \didN{} observations: \treatN{} (\treatPct{} percent) treatment and \controlN{} (\controlPct{} percent) control. The treatment group comprises \treatOccN{} distinct occupations and the control group \controlOccN{}, providing \treatOccN{} and \controlOccN{} effective clusters for inference, respectively.\footnote{The five largest occupations by sample size in the treatment group are: \topFiveTreat{}. In the control group: \topFiveCtrl{}. Appendix Table~\ref{tab:occlist} provides the complete occupation listing with treatment status, sample size, and telework feasibility scores.}


% ==============================================================================
\section{Results}
\label{sec:results}
% ==============================================================================

\subsection{Descriptive Evidence}

Figure~\ref{fig:trends} plots mean real hourly wages by task-telework quadrant from 2017Q1 through 2023Q1. There is a clear hierarchy: high-abstract non-teleworkable occupations earn the highest wages, followed by high-abstract teleworkable, then low-abstract teleworkable, and finally low-abstract non-teleworkable. The two high-abstract groups track each other closely in the pre-period but begin to diverge after 2020, with teleworkable occupations pulling ahead. This divergence is not apparent among low-abstract occupations. Figure~\ref{fig:didtrends} focuses on the DID estimation sample, plotting mean log wages for treatment and control groups. The two series move in near-lockstep from 2017 through 2019, then diverge from 2020Q1 onward, consistent with the parallel trends assumption holding in the pre-period.

\begin{figure}[H]
\centering
\caption{Mean Real Hourly Wages by Task-Telework Quadrant}
\label{fig:trends}
\includegraphics[width=0.75\textwidth]{fig1_wage_trends_quadrant.pdf}

\bigskip
\begin{minipage}{0.95\textwidth}
\small\textit{Notes:} The data are from CPS-ORG, 2017Q1--2023Q1. This figure shows mean real hourly wages (2019 dollars) by task-telework quadrant. Occupations are classified by O*NET abstract-task intensity (top tercile cutoff) and Dingel-Neiman telework feasibility. Vertical dashed line marks 2020Q1.
\end{minipage}
\end{figure}

\begin{figure}[H]
\centering
\caption{Mean Log Wages: Treatment vs.\ Control (High-Abstract Occupations)}
\label{fig:didtrends}
\includegraphics[width=0.75\textwidth]{fig1b_did_trends.pdf}

\bigskip
\begin{minipage}{0.95\textwidth}
\small\textit{Notes:} The data are from CPS-ORG, 2017Q1--2023Q1. This figure shows mean log real hourly wages for the DID estimation sample restricted to high-abstract occupations (top tercile). In blue is the treatment group (teleworkable, Dingel-Neiman = 1). In red is the control group (non-teleworkable). Vertical dashed line marks 2020Q1.
\end{minipage}
\end{figure}

Using the CPS telework supplement (available from October 2022), I verify first-stage relevance: \fsTreatRate{} percent of workers in teleworkable high-abstract occupations reported teleworking for pay, compared to \fsControlRate{} percent in non-teleworkable high-abstract occupations, a \fsGap{} percentage point gap confirming the Dingel-Neiman classification predicts realized telework adoption. A limitation is that the CPS telework supplement begins only in October 2022, late in the sample period. The first-stage gap may have been larger during 2020--2021 when lockdowns constrained on-site work, and may have narrowed as return-to-office policies took hold. The 2022--2023 gap likely represents a lower bound on average first-stage separation over the post-treatment window, but I cannot verify this directly. Figure~\ref{fig:firststage} displays telework adoption rates across all four quadrants.

\begin{figure}[H]
\centering
\caption{First-Stage Evidence: Actual Telework Adoption by Quadrant}
\label{fig:firststage}
\includegraphics[width=0.75\textwidth]{fig4_first_stage.pdf}

\bigskip
\begin{minipage}{0.95\textwidth}
\small\textit{Notes:} The data are from CPS TELWRKPAY variable, October 2022 onward. This figure shows the share of workers who report teleworking for pay in the previous week, weighted by CPS earnings weight.
\end{minipage}
\end{figure}


\subsection{Baseline Difference-in-Differences}
\label{sec:baseline}

Table~\ref{tab:baseline} presents the baseline DID results across seven progressively saturated specifications. The coefficient on Treated $\times$ Post is remarkably stable. Column (1) includes only occupation and year-quarter fixed effects and yields $\hat{\beta} = \mOneCoef$ ($\mainPval$). Adding individual demographic controls (column 2), education and race FE (column 3), industry FE (column 4), and state FE (column 5, the preferred specification) leaves the estimate essentially unchanged at \specMinPct--\specMaxPct{} percent. Column (6) replaces CPI-adjusted wages with RPP-adjusted wages; the coefficient is virtually identical at \mSixCoef. Column (7) introduces industry $\times$ year-quarter FE, absorbing all sector-specific time-varying shocks. The estimate attenuates to \mSevenCoef{} but remains statistically significant ($\mSevenPval$), providing a credible lower bound.

The stability across specifications is noteworthy. The $R^2$ increases from \rSqMin{} to \rSqMax{} as controls are added, but the treatment effect barely moves. As a formal coefficient stability test, I compute the \citet{oster2019} bound parameter $\delta$, using the restricted model (column 1, occupation and year-quarter FE only) and the preferred specification (column 5) with $R_{\max} = \min(1.3 \times \tilde{R}^2, 1) = \osterRmax$ following Oster's recommended rule of thumb. Because the coefficient is virtually unchanged between the restricted (\betaRestricted) and full (\betaFull) specifications while $R^2$ increases from \rSqRestricted{} to \rSqFull{}, the implied $\delta = \osterDelta$, meaning unobservable selection would need to be far more important than the combined observable controls to explain away the result. The implied bias-adjusted treatment effect under $\delta = 1$ and this $R_{\max}$ is \biasAdjBetaPct{} percent, virtually identical to the OLS estimate. A caveat is warranted: because treatment is assigned at the occupation level and occupation fixed effects are already included in the restricted model, the individual-level controls added in subsequent specifications (demographics, education, industry, state) are largely orthogonal to the treatment variable. The coefficient stability and large $\delta$ are therefore partly mechanical and should not be interpreted as strong evidence against all forms of unobservable confounding---only against confounders that operate through the observable channels being progressively added. Figure~\ref{fig:progressive} visualizes this stability.

\begin{table}[H]
\centering
\caption{Baseline Difference-in-Differences: Effect of Telework Access on Abstract-Task Wages}
\label{tab:baseline}
\footnotesize
\begin{tabular}{lccccccc}
\toprule
 & (1) & (2) & (3) & (4) & (5) & (6) & (7) \\
\midrule
Treated $\times$ Post & $0.023^{**}$ & $0.022^{**}$ & $0.020^{**}$ & $0.022^{**}$ & $0.023^{**}$ & $0.022^{**}$ & $0.018^{**}$ \\
 & (0.010) & (0.009) & (0.009) & (0.009) & (0.009) & (0.009) & (0.009) \\[6pt]
Occupation FE & X & X & X & X & X & X & X \\
Year-Quarter FE & X & X & X & X & X & X & X \\
Demographics &  & X & X & X & X & X & X \\
Education, Race FE &  &  & X & X & X & X & X \\
Industry FE &  &  &  & X & X & X &  \\
State FE &  &  &  &  & X & X & X \\
Ind $\times$ YQ FE &  &  &  &  &  &  & X \\
\midrule
Num.\ Obs. & 94,398 & 94,398 & 94,398 & 94,398 & 94,398 & 94,398 & 94,398 \\
$R^2$ & 0.392 & 0.443 & 0.489 & 0.501 & 0.541 & 0.529 & 0.544 \\
\bottomrule
\multicolumn{8}{l}{\footnotesize $^{***}p < 0.01$; $^{**}p < 0.05$; $^{*}p < 0.1$}
\end{tabular}

\bigskip
\begin{minipage}{0.95\textwidth}
\small\textit{Notes:} All regressions use CPS-ORG data, 2017Q1--2023Q1. Sample restricted to high-abstract occupations (top tercile of O*NET abstract-task index). The dependent variable is log real hourly wages (CPI-U-RS adjusted to 2019 dollars). Standard errors clustered at the occupation level (185 clusters) in parentheses. Column (5) is the preferred specification. Column (6) uses RPP-adjusted wages. Column (7) replaces industry FE with industry $\times$ year-quarter FE to absorb sector-specific pandemic shocks.
\end{minipage}
\end{table}


\begin{figure}[H]
\centering
\caption{Stability of the DID Estimate Across Specifications}
\label{fig:progressive}
\includegraphics[width=0.75\textwidth]{fig8_progressive_controls.pdf}

\bigskip
\begin{minipage}{0.95\textwidth}
\small\textit{Notes:} The data are from CPS-ORG, 2017Q1--2023Q1. Each point shows the coefficient on Treated $\times$ Post from a different specification, with 95 percent confidence intervals. Specifications progressively add controls from left to right.
\end{minipage}
\end{figure}


\subsection{Event-Study Estimates}
\label{sec:eventstudy}

Figure~\ref{fig:eventstudy} presents the event-study estimates from equation~(\ref{eq:eventstudy}). The pre-treatment coefficients ($k = -12$ through $k = -2$) are tightly clustered around zero with no discernible trend, providing strong support for the parallel trends assumption. A joint F-test for all pre-period coefficients yields $p = \preTrendP$, failing to reject the null of zero pre-trends by a wide margin.

The post-treatment coefficients show gradual divergence. The initial quarters (2020Q1--Q2) show small, imprecise effects, consistent with the immediate lockdown creating disruptions across all occupations. By 2020Q3--Q4, the treatment effect begins to emerge, and by 2021--2022 the coefficients stabilize in the range of 2--4 percent. The average of the post-treatment event-study coefficients is \avgPostESPct{} percent, closely matching the baseline DID estimate and confirming that the static specification adequately summarizes the dynamic treatment path. This dynamic pattern is consistent with equilibrium adjustment: as telework arrangements became permanent features of organizational design rather than emergency measures, the wage effects crystallized.

\begin{figure}[H]
\centering
\caption{Event Study: Effect of Telework Access on Abstract-Task Wages}
\label{fig:eventstudy}
\includegraphics[width=0.75\textwidth]{fig2_event_study.pdf}

\bigskip
\begin{minipage}{0.95\textwidth}
\small\textit{Notes:} The data are from CPS-ORG, 2017Q1--2023Q1. This figure shows coefficients from equation~(\ref{eq:eventstudy}) with 95 percent confidence intervals. Reference period: 2019Q4. Blue shading indicates post-treatment. All specifications include occupation, year-quarter, education, race, industry, and state FE. Standard errors clustered at the occupation level.
\end{minipage}
\end{figure}

Figure~\ref{fig:overlay} overlays the baseline event study with the saturated specification (adding industry $\times$ year-quarter FE). The pre-trends remain flat in both. The post-treatment coefficients are attenuated in the saturated specification, consistent with some of the effect operating through industry-level channels, but the broad pattern of gradual divergence is preserved. The joint F-test for pre-trends in the saturated specification yields $p = \satPreTrendP$.

\begin{figure}[H]
\centering
\caption{Event Study: Baseline vs.\ Saturated Specification}
\label{fig:overlay}
\includegraphics[width=0.75\textwidth]{fig9_event_study_overlay.pdf}

\bigskip
\begin{minipage}{0.95\textwidth}
\small\textit{Notes:} The data are from CPS-ORG, 2017Q1--2023Q1. Both specifications include occupation, year-quarter, education, race, and state FE. In blue is the baseline specification. In orange is the saturated specification, which additionally includes industry $\times$ year-quarter FE. Standard errors clustered at the occupation level.
\end{minipage}
\end{figure}


\subsection{Continuous Treatment and Triple Difference}
\label{sec:continuous}

Table~\ref{tab:continuous} reports results from the continuous treatment specification and the triple-difference. The continuous treatment DID within the high-abstract sample yields $\hat{\beta} = \contCoef$ ($\contPval$): a one-unit increase in telework feasibility (from fully non-teleworkable to fully teleworkable) leads to a \contCoefPct{} percent wage premium post-pandemic. This provides a dose-response interpretation consistent with the binary specification. The triple-difference, estimated on the full sample (all occupations), yields \tripleCoef{} on the Abstract $\times$ Teleworkable $\times$ Post interaction, confirming the telework premium is specific to abstract-task intensive occupations.

To assess whether the dose-response relationship is approximately linear, I split the Dingel-Neiman feasibility score into terciles within the high-abstract sample and interact each tercile indicator with the post-treatment dummy. Relative to the lowest-feasibility tercile, the middle tercile shows a coefficient of \tercileMidCoef{} (SE = \tercileMidSE) and the highest tercile shows \tercileHighCoef{} (SE = \tercileHighSE). Both are positive and statistically significant, confirming a dose-response relationship, though the coefficients are similar in magnitude. This pattern is consistent with a threshold effect: once telework feasibility crosses a moderate level, additional feasibility does not proportionally increase the wage premium, suggesting the mechanism operates through access to telework rather than intensity of telework.

Figure~\ref{fig:continuous_es} presents the continuous treatment event study, showing the same pattern of flat pre-trends and gradual post-treatment divergence.

\begin{table}[H]
\centering
\caption{Continuous Treatment and Triple-Difference Estimates}
\label{tab:continuous}
\footnotesize
\resizebox{\textwidth}{!}{%
\begin{tabular}{lccc}
\toprule
& (1) & (2) & (3) \\
& Continuous DID & Triple-Difference & Binned Feasibility \\
& (High-abstract sample) & (Full sample) & (High-abstract sample) \\
\midrule
TW Feasibility $\times$ Post & $\contCoef\contStars$ & & \\
& (\contSE) & & \\[6pt]
Abstract $\times$ TW Feas.\ $\times$ Post & & $\tripleCoef\tripleStars$ & \\
& & (\tripleSE) & \\[6pt]
Mid Tercile $\times$ Post & & & $\tercileMidCoef\tercileMidStars$ \\
& & & (\tercileMidSE) \\[6pt]
High Tercile $\times$ Post & & & $\tercileHighCoef\tercileHighStars$ \\
& & & (\tercileHighSE) \\[6pt]
\midrule
Num.\ Obs. & \didN & \tripleN & \tercileN \\
$R^2$ & & & \tercileRsq \\
FE & \multicolumn{3}{c}{Occ, YQ, Educ, Race, Ind, St} \\
\bottomrule
\multicolumn{4}{l}{\footnotesize $^{***}p < 0.01$; $^{**}p < 0.05$; $^{*}p < 0.1$}
\end{tabular}%
}

\bigskip
\begin{minipage}{0.95\textwidth}
\small\textit{Notes:} All regressions use CPS-ORG data, 2017Q1--2023Q1. Standard errors clustered at the occupation level (\nOccClusters{} clusters) in parentheses. TW Feasibility is the continuous Dingel-Neiman score $\in [0,1]$. Column (3) splits the feasibility score into terciles within the high-abstract sample; the omitted category is the lowest-feasibility tercile. All specifications include individual demographic controls.
\end{minipage}
\end{table}

\begin{figure}[H]
\centering
\caption{Continuous Treatment Event Study}
\label{fig:continuous_es}
\includegraphics[width=0.75\textwidth]{fig3_continuous_event_study.pdf}

\bigskip
\begin{minipage}{0.95\textwidth}
\small\textit{Notes:} The data are from CPS-ORG, 2017Q1--2023Q1. This figure shows the marginal effect of a one-unit change in telework feasibility at each quarter. Reference period: 2019Q4. Green shading indicates post-treatment.
\end{minipage}
\end{figure}


\subsection{Heterogeneity}
\label{sec:heterogeneity}

Table~\ref{tab:het} and Figure~\ref{fig:het} present split-sample estimates across four dimensions. The most striking result is geographic: in metropolitan areas, the treatment effect is \metroCoefPct{} percent, statistically significant ($\metroPval$); in non-metropolitan areas, the point estimate is indistinguishable from zero ($\nonmetroPval$). This is consistent with the monopsony-reduction mechanism (H2): telework expands effective labor markets most where employer concentration constrains wages. Figure~\ref{fig:metro_es} in the Appendix presents separate event studies for metropolitan and non-metropolitan subsamples, confirming that the divergence concentrates in urban labor markets while non-metropolitan areas show flat treatment-control dynamics throughout. The metro subsample shows flat pre-trends followed by clear post-2020 divergence, while the non-metro subsample shows no discernible pattern in either period, with wide confidence intervals reflecting the smaller sample.

The effect is somewhat larger for women (\femaleCoefPct{} percent, $\femalePval$) than for men (\maleCoefPct{} percent, $\malePval$), consistent with women placing higher value on telework flexibility \citep{mas2017, wiswall2018}. Workers with a bachelor's degree or higher show a coefficient of \baCoefPct{} percent compared to \nobaCoefPct{} percent for those without, suggesting the premium operates across the education distribution within high-abstract occupations. Union members exhibit the largest effect at \unionCoefPct{} percent ($\unionPval$, $N = \unionN$), compared to \nonunionCoefPct{} percent for non-union workers ($\nonunionPval$, $N = \nonunionN$), potentially reflecting union bargaining incorporating telework-related productivity gains into wage contracts more effectively than individual bargaining. The union subsample is substantially smaller, limiting statistical power for the equality test. Formal Wald tests of coefficient equality show that none of the four subgroup differences reach conventional significance levels: metro/non-metro ($z = \metroEqZ$, $p = \metroEqP$), gender ($z = \genderEqZ$, $p = \genderEqP$), education ($z = \educEqZ$, $p = \educEqP$), and union ($z = \unionEqZ$, $p = \unionEqP$). The metro and union splits approach marginal significance, consistent with economically meaningful heterogeneity that the split samples lack power to detect. Splitting by age reveals similar effects across career stages: the treatment effect is \ageUnderFortyCoef{} percent for workers under 40 ($\ageUnderFortyPval$) and \ageFortyplusCoef{} percent for workers 40 and older ($\ageFortyplusPval$), with no significant difference ($z = \ageEqZ$, $p = \ageEqP$). The absence of age heterogeneity suggests the premium is broadly based across career stages rather than concentrated among workers with particular demographic profiles.

Because the Dingel-Neiman classification captures telework feasibility rather than actual adoption, the baseline estimate is an intent-to-treat (ITT) effect. I emphasize that the ITT is the policy-relevant parameter: it captures the wage consequences of being in an occupation where telework became feasible, regardless of individual adoption. A Wald rescaling to a LATE is not informative here: the first-stage gap is modest (\fsGapRaw{} percentage points), \fsControlTelework{} percent of control occupations also report teleworking (substantial non-compliance), and the thin first stage produces a mechanically inflated ratio.

\begin{table}[H]
\centering
\caption{Heterogeneity in the Telework-Wage Effect}
\label{tab:het}
\scriptsize
\resizebox{\textwidth}{!}{%
\begin{tabular}{lcccccccccc}
\toprule
 & \multicolumn{2}{c}{Gender} & \multicolumn{2}{c}{Geography} & \multicolumn{2}{c}{Education} & \multicolumn{2}{c}{Union Status} & \multicolumn{2}{c}{Age} \\
\cmidrule(lr){2-3} \cmidrule(lr){4-5} \cmidrule(lr){6-7} \cmidrule(lr){8-9} \cmidrule(lr){10-11}
 & Male & Female & Metro & Non-Metro & BA+ & No BA & Union & Non-Union & $<$40 & $\geq$40 \\
 & (1) & (2) & (3) & (4) & (5) & (6) & (7) & (8) & (9) & (10) \\
\midrule
Treated $\times$ Post & $0.020$ & $0.024^{**}$ & $0.027^{***}$ & $0.000$ & $0.028^{**}$ & $0.021^{*}$ & $0.042^{***}$ & $0.016^{*}$ & $0.018^{*}$ & $0.023^{**}$ \\
 & (0.014) & (0.011) & (0.010) & (0.015) & (0.012) & (0.011) & (0.014) & (0.009) & (0.009) & (0.011) \\[6pt]
\midrule
Num.\ Obs. & 37,806 & 56,589 & 75,487 & 17,878 & 36,105 & 58,287 & 11,754 & 81,104 & 48,985 & 45,413 \\
$R^2$ & 0.501 & 0.576 & 0.543 & 0.564 & 0.462 & 0.503 & 0.574 & 0.530 & 0.557 & 0.518 \\
\bottomrule
\multicolumn{11}{l}{\footnotesize $^{***}p < 0.01$; $^{**}p < 0.05$; $^{*}p < 0.1$}
\end{tabular}%
}

\bigskip
\begin{minipage}{0.95\textwidth}
\small\textit{Notes:} All regressions use CPS-ORG data, 2017Q1--2023Q1. Each column reports the coefficient on Treated $\times$ Post from a separate regression on the indicated subsample. All specifications include occupation, year-quarter, education (where applicable), race, industry, and state fixed effects plus individual demographic controls. Standard errors clustered at the occupation level in parentheses.
\end{minipage}
\end{table}


\begin{figure}[H]
\centering
\caption{Heterogeneity in the Telework-Wage Effect}
\label{fig:het}
\includegraphics[width=0.75\textwidth]{fig5_heterogeneity.pdf}

\bigskip
\begin{minipage}{0.95\textwidth}
\small\textit{Notes:} The data are from CPS-ORG, 2017Q1--2023Q1. Each point is from a separate regression on the indicated subgroup, with 95 percent confidence intervals. All specifications include occupation, year-quarter, education, race, industry, and state FE.
\end{minipage}
\end{figure}


% ==============================================================================
\section{Robustness and Sensitivity}
\label{sec:robustness}
% ==============================================================================

Table~\ref{tab:robust} and Figure~\ref{fig:robust} summarize results from the robustness specifications. A placebo test applying the DID to pre-pandemic data (2017--2019) with a fake treatment at 2018Q1 yields \placeboCoef{} ($p = \placeboPval$), confirming the treatment effect is not an artifact of pre-existing differential trends. Dropping the 2020Q2--Q4 quarters (the acute lockdown period) yields \noPeakCoef{} ($\noPeakPval$), suggesting the result is not driven by compositional disruptions during the pandemic peak. Using the top quartile (rather than top tercile) of abstract-task intensity yields \topQuartCoef{} ($\topQuartPval$), while the top 40 percent yields \topFortyCoef{} ($\topFortyPval$). Double-clustering standard errors by occupation and state yields virtually identical significance. Unweighted regression yields \unwgtCoef{} ($\unwgtPval$).

A concern with the control group is that healthcare practitioners (SOC major group 29), including surgeons, dentists, and veterinarians, may have wages determined by institutional or regulatory factors rather than competitive labor market forces. Healthcare occupations also experienced distinctive pandemic demand shocks: elective procedure bans suppressed revenue and potentially wages in early 2020, followed by a strong rebound in 2021--2022 driven by pent-up demand. These shocks do not map cleanly onto industry-by-time fixed effects because healthcare wage determination reflects fee schedules, insurance reimbursement rates, and facility-specific revenue dynamics that vary within industry codes. If the healthcare rebound artificially depressed control-group wage growth relative to counterfactual, the treatment effect would be overstated. Excluding all SOC major group 29 occupations from the control group yields a coefficient of \noHealthCoef{} ($\noHealthPval$, $N = \noHealthN$). The attenuation from \mainCoef{} to \noHealthCoef{} likely reflects reduced statistical power rather than a substantive change in the estimate: excluding SOC major group 29 removes approximately 12,000 observations from the control group, reducing the effective cluster count. The point estimate remains within the \specMinPct--\specMaxPct{} percent range of the main specifications, confirming that the result is not driven by the healthcare wage structure.

Restricting the sample to prime-age workers (25--54) addresses concerns about differential retirement or labor force entry patterns during the pandemic. The preferred specification on this subsample yields \primeAgeCoef{} ($\primeAgePval$, $N = \primeAgeN$), within the range of the main estimates.

Perhaps the most important robustness check concerns the tercile cutoff itself. The two largest treatment occupations by sample size---customer service representatives and personal care aides---have abstract-task scores of 0.12 and 0.16, respectively, well below the treatment-group mean. If the result were driven by these borderline occupations (whose telework feasibility may proxy for something other than abstract-task content), the estimate would weaken substantially when they are dropped. Restricting the sample to occupations above the median abstract-task score within the top tercile eliminates all such borderline cases and yields \aboveMedAbsCoef{} ($\aboveMedAbsPval$, $N = \aboveMedAbsN$). This estimate is squarely within the \specMinPct--\specMaxPct{} percent range of the main specifications and, if anything, slightly larger in magnitude, confirming that borderline occupations attenuate rather than inflate the baseline estimate.

\begin{table}[H]
\centering
\caption{Robustness Checks}
\label{tab:robust}
\scriptsize
\resizebox{\textwidth}{!}{%
\begin{tabular}{lccccccccccc}
\toprule
 & Preferred & Placebo & No 2020 & Top 25\% & Top 40\% & Dbl.\ Clust. & Unwgt. & Ind$\times$YQ & No HC & Prime & High Abs. \\
 & (1) & (2) & (3) & (4) & (5) & (6) & (7) & (8) & (9) & (10) & (11) \\
\midrule
Treated $\times$ Post & $0.023^{**}$ &  & $0.024^{***}$ & $0.018^{*}$ & $0.025^{***}$ & $0.023^{***}$ & $0.016^{**}$ & $0.018^{**}$ & $0.019^{*}$ & $0.018^{**}$ & $0.028^{**}$ \\
 & (0.009) &  & (0.009) & (0.010) & (0.008) & (0.008) & (0.008) & (0.009) & (0.010) & (0.009) & (0.013) \\[6pt]
Placebo $\times$ Post &  & $0.004$ &  &  &  &  &  &  &  &  &  \\
 &  & (0.008) &  &  &  &  &  &  &  &  &  \\[6pt]
\midrule
Num.\ Obs. & 94,398 & 49,245 & 85,269 & 70,136 & 112,561 & 94,398 & 94,398 & 94,398 & 82,373 & 66,521 & 46,972 \\
$R^2$ & 0.541 & 0.554 & 0.542 & 0.517 & 0.530 & 0.541 & 0.545 & 0.544 & 0.484 & 0.518 & 0.533 \\
\bottomrule
\multicolumn{12}{l}{\footnotesize $^{***}p < 0.01$; $^{**}p < 0.05$; $^{*}p < 0.1$}
\end{tabular}%
}

\bigskip
\begin{minipage}{0.95\textwidth}
\small\textit{Notes:} All regressions use CPS-ORG data. Standard errors clustered at the occupation level in parentheses. Column (1) reproduces the preferred specification from Table~\ref{tab:baseline}. Column (2) uses pre-pandemic data only (2017--2019) with a placebo treatment at 2018Q1. Column (3) excludes 2020Q2--Q4. Columns (4)--(5) use alternative abstract-task thresholds. Column (6) double-clusters by occupation and state. Column (7) is unweighted. Column (8) includes industry $\times$ year-quarter FE. Column (9) excludes healthcare practitioners (SOC major group 29) from the control group. Column (10) restricts to prime-age workers (25--54). Column (11) restricts to occupations above the median abstract-task score within the top tercile.
\end{minipage}
\end{table}


\begin{figure}[H]
\centering
\caption{Robustness of the Baseline DID Estimate}
\label{fig:robust}
\includegraphics[width=0.75\textwidth]{fig6_robustness.pdf}

\bigskip
\begin{minipage}{0.95\textwidth}
\small\textit{Notes:} The data are from CPS-ORG, 2017Q1--2023Q1. Each point shows the coefficient on Treated $\times$ Post from a different specification, with 95 percent confidence intervals. Shaded band shows the 95 percent CI of the preferred specification. The placebo test uses pre-pandemic data with a fake treatment at 2018Q1.
\end{minipage}
\end{figure}

Figure~\ref{fig:density} plots kernel densities of log wages for treatment and control groups before and after the pandemic. The post-period density for the treatment group shifts rightward relative to the pre-period, while the control group shows a smaller shift, consistent with the DID estimate capturing a genuine shift in the conditional wage distribution rather than an artifact of outliers or distributional shape changes.

\begin{figure}[H]
\centering
\caption{Log Wage Distribution: Treatment vs.\ Control, Pre vs.\ Post}
\label{fig:density}
\includegraphics[width=0.75\textwidth]{fig7_wage_density.pdf}

\bigskip
\begin{minipage}{0.95\textwidth}
\small\textit{Notes:} The data are from CPS-ORG, 2017Q1--2023Q1. This figure shows kernel density estimates, weighted by CPS earnings weight. On the left is the teleworkable (treatment) group. On the right is the non-teleworkable (control) group.
\end{minipage}
\end{figure}

I implement the \citet{rambachan2023} sensitivity analysis (Honest DID) to assess robustness to violations of the parallel trends assumption. The relative magnitudes restriction assumes the maximum post-treatment violation is at most $\bar{M}$ times the maximum pre-treatment violation. At $\bar{M} = 0$ (standard parallel trends), the 95 percent CI for the first post-treatment period is $[\honestRmLbZero, \honestRmUbZero]$, and the average post-treatment effect interval is $[\honestRmAvgLb, \honestRmAvgUb]$. Under the relative magnitudes approach, the results remain informative for moderate values of $\bar{M}$, excluding zero up to approximately $\bar{M} = 1.5$---that is, the conclusion survives even if post-treatment trend violations are 50 percent larger than the maximum observed pre-treatment deviation.

The smoothness restriction ($\Delta^{SD}$), which bounds the second derivative of the violation path, yields a CI of $[\honestSdLbZero, \honestSdUbZero]$ at $M = 0$ (linear extrapolation of pre-trends), which includes zero. This divergence between the two approaches warrants careful discussion. The $\Delta^{SD}$ approach extrapolates a linear trend from the pre-period forward; when pre-treatment coefficients are very close to zero (as they are here---the joint F-test yields $p = \preTrendP$), even tiny deviations from exact linearity generate a non-trivial extrapolated violation that quickly widens confidence intervals for moderate-sized treatment effects. \citet{rambachan2023} note that the two restrictions answer different questions: $\Delta^{RM}$ asks whether post-treatment violations of the kind observed pre-treatment could explain the result, while $\Delta^{SD}$ asks whether a smooth extrapolation of pre-trends could do so. For this application, the relative magnitudes approach may be more informative because it directly benchmarks against observed pre-period behavior. However, the fact that the effect does not survive the smoothness restriction is a genuine limitation. The result should be interpreted as robust to moderate violations of parallel trends under the relative magnitudes framework, but fragile under the smoothness framework---a distinction that readers should weigh based on which assumption they find more plausible for this setting (Figure~\ref{fig:honest}).

\begin{figure}[H]
\centering
\caption{Honest DID Sensitivity Analysis \citep{rambachan2023}}
\label{fig:honest}

\begin{subfigure}[t]{0.48\textwidth}
    \centering
    \includegraphics[width=\textwidth]{fig10a_honest_did_rm.pdf}
    \caption{Relative Magnitudes}
    \label{fig:honestRM}
\end{subfigure}
\hfill
\begin{subfigure}[t]{0.48\textwidth}
    \centering
    \includegraphics[width=\textwidth]{fig10b_honest_did_sd.pdf}
    \caption{Smoothness Restriction}
    \label{fig:honestSD}
\end{subfigure}

\bigskip
\begin{minipage}{0.95\textwidth}
\small\textit{Notes:} The data are from CPS-ORG, 2017Q1--2023Q1. On the left, robust CIs under $\Delta^{RM}(\bar{M})$, bounding the max post-treatment violation relative to max pre-treatment violation. On the right, robust CIs under $\Delta^{SD}(M)$, bounding the change in slope of the violation path. Both panels show 95 percent CIs for the first post-treatment period.
\end{minipage}
\end{figure}


% ==============================================================================
\section{Conclusion}
\label{sec:conclusion}
% ==============================================================================

Teleworkable abstract-task occupations experienced a \mainCoefPct{} percent relative wage premium following the pandemic-induced telework expansion, translating to roughly \$\premHourLo--\$\premHourHi{} per hour, or \$\premAnnualLo--\$\premAnnualHi{} per year for a worker earning the sample mean of approximately \$\meanWage{} per hour. This estimate is stable across seven progressively saturated specifications (\specMinPct--\specMaxPct{} percent), supported by flat pre-trends, a null placebo test, and consistent continuous-treatment and triple-difference results. The positive sign indicates that channels increasing wages (productivity gains and monopsony reduction) dominate channels decreasing wages (supply expansion and compensating differentials).

Several limitations warrant discussion. The sample window ends at 2023Q1, and longer-run effects may differ as return-to-office policies and competitive dynamics evolve. The Dingel-Neiman classification captures feasibility, not actual adoption; the treatment is intent-to-treat and may understate the effect of actual telework. SUTVA violations are possible through general equilibrium effects, and as discussed in Section~\ref{sec:design}, the direction of any resulting bias is ambiguous. The Honest DID analysis reveals that under the smoothness restriction the CI includes zero, though the more economically motivated relative magnitudes approach confirms robustness for plausible violation magnitudes. This underscores that moderate-sized effects in DID designs require careful attention to parallel trends credibility and to the choice of sensitivity framework.

The heterogeneity results help distinguish between competing mechanisms. The metro\slash non-metro split strongly supports the monopsony-reduction channel: telework reduces effective labor market concentration in dense urban areas where commuting constraints previously limited workers' outside options. The union premium (\unionCoefPct{} versus \nonunionCoefPct{} percent) suggests institutions matter for translating telework-related productivity gains into wages.

These findings interact in an informative way with the concurrent literature on post-pandemic wage compression. \citet{autor2023} document that the post-2020 labor market compressed the overall wage distribution, with workers at the 10th percentile experiencing faster real wage growth than those at the 90th percentile---a reversal of four decades of widening inequality. \citet{carroll2025} confirm this compression pattern using CPS data over a similar window. My results are not contradictory but complementary: the post-COVID labor market simultaneously compressed wages \textit{across} the skill distribution (through tightening of the low-wage labor market and enhanced outside options for non-teleworkable service workers) while creating new inequalities \textit{within} skill groups at the top (through telework-mediated monopsony reduction for abstract-task workers). The net effect on upper-tail inequality depends on which force dominates. The \specMinPct--\specMaxPct{} percent within-abstract-task premium I estimate is modest relative to the 7--8 percent compression at the 10th-vs-90th percentile documented by \citeauthor{autor2023}, suggesting that the compression force dominated overall but that telework access opened a new margin of within-group inequality that existing frameworks---which treat all abstract-task workers symmetrically---do not capture.

These findings have direct implications for return-to-office (RTO) policies. If telework access raises wages through reduced monopsony power, as the metropolitan concentration of effects suggests, mandating physical presence could restore employer wage-setting power and compress abstract-task wages. Recent high-profile mandates, including Amazon's five-day return-to-office policy announced in September 2024, underscore the relevance of these findings: if the \specMinPct--\specMaxPct{} percent premium reflects reduced monopsony power, full RTO mandates may partially reverse this wage gain. Firms designing hybrid work policies should recognize that the \specMinPct--\specMaxPct{} percent premium represents the market's partial pricing of remote work flexibility, a benchmark for calibrating the productivity-flexibility tradeoff. The null effect in non-metropolitan areas suggests the telework revolution has not yet equalized labor market opportunities across space, at least within the abstract-task segment.

Future work should extend the analysis as longer CPS-ORG data become available, enabling assessment of whether the premium persists or dissipates as remote work norms solidify, and investigate firm-level mechanisms using matched employer-employee data to disentangle the productivity, monopsony-reduction, and compensating differential channels.


% ==============================================================================
% Data Availability
% ==============================================================================
\subsection*{Data Availability Statement}
The Current Population Survey Outgoing Rotation Group (CPS-ORG) microdata are available through IPUMS CPS (\url{https://cps.ipums.org}). O*NET task content data are available from the National Center for O*NET Development (\url{https://www.onetcenter.org}). The Dingel-Neiman telework feasibility classification is available at \url{https://github.com/jdingel/DingelNeiman-workathome}. Bureau of Economic Analysis Regional Price Parities are available at \url{https://www.bea.gov}. Replication code is available from the author upon request.

% References
% ==============================================================================
\newpage
\singlespacing
\bibliography{references}


% ==============================================================================
% Appendix
% ==============================================================================
\newpage
\onehalfspacing
\appendix

\section{Additional Tables}

\footnotesize
\begin{longtable}{p{5.5cm}lrllr}
\caption{Occupations in the Estimation Sample} \label{tab:occlist} \\
\toprule
Occupation & SOC & N & Group & Abstract & TW Score \\
\midrule
\endfirsthead
\multicolumn{6}{l}{\textit{Table~\ref{tab:occlist} continued}} \\
\toprule
Occupation & SOC & N & Group & Abstract & TW Score \\
\midrule
\endhead
\midrule
\multicolumn{6}{r}{\textit{Continued on next page}} \\
\endfoot
\bottomrule
\endlastfoot
Customer Service Representatives & 43-4051 & 7,141 & Treatment & 0.12 & 0.61 \\
Personal Care Aides & 39-9021 & 4,446 & Treatment & 0.16 & 0.75 \\
Teacher Assistants & 25-9041 & 3,325 & Treatment & 0.85 & 0.75 \\
Managers, All Other & 11-9199 & 3,323 & Treatment & 1.05 & 0.86 \\
First-Line Supervisors of Office and Adm & 43-1011 & 2,512 & Treatment & 0.49 & 1.00 \\
Accountants and Auditors & 13-2011 & 1,827 & Treatment & 0.81 & 1.00 \\
Other Teachers and Instructors & 25-3000 & 1,708 & Treatment & 0.83 & 1.00 \\
Preschool and Kindergarten Teachers & 25-2010 & 1,670 & Treatment & 0.72 & 0.87 \\
Elementary and Middle School Teachers & 25-2020 & 1,437 & Treatment & 0.72 & 0.87 \\
Other Office/Admin Support Workers & 43-9199 & 1,410 & Treatment & 0.86 & 1.00 \\
Designers & 27-1020 & 1,260 & Treatment & 0.23 & 0.82 \\
Human Resources Workers & 13-1070 & 1,048 & Treatment & 0.22 & 0.87 \\
Financial Managers & 11-3031 & 1,035 & Treatment & 0.99 & 1.00 \\
First-Line Supervisors of Non-Retail Sal & 41-1012 & 959 & Treatment & 0.26 & 1.00 \\
Residential Advisors & 39-9030 & 857 & Treatment & 0.16 & 0.75 \\
Medical and Health Services Managers & 11-9111 & 828 & Treatment & 1.38 & 1.00 \\
General and Operations Managers & 11-1021 & 774 & Treatment & 1.02 & 1.00 \\
Postsecondary Teachers & 25-1000 & 652 & Treatment & 1.42 & 0.96 \\
Marketing and Sales Managers & 11-2020 & 616 & Treatment & 0.93 & 1.00 \\
Tax Preparers/Financial Specialists & 13-2070 & 615 & Treatment & 0.60 & 0.75 \\
Business Operations Specialists, All Oth & 13-1199 & 584 & Treatment & 0.61 & 0.85 \\
Property, Real Estate, and Community Ass & 11-9141 & 566 & Treatment & 0.24 & 1.00 \\
Supervisors, Transportation Workers & 53-1000 & 562 & Treatment & 0.16 & 0.79 \\
Interviewers, Except Eligibility and Loa & 43-4111 & 550 & Treatment & 0.80 & 1.00 \\
Claims Adjusters and Examiners & 13-1030 & 541 & Treatment & 0.22 & 0.87 \\
Management Analysts & 13-1111 & 539 & Treatment & 1.92 & 1.00 \\
Civil Engineers & 17-2051 & 501 & Treatment & 0.73 & 1.00 \\
Social and Community Service Managers & 11-9151 & 497 & Treatment & 0.63 & 1.00 \\
Sales Representatives, Services, All Oth & 41-3099 & 475 & Treatment & 0.32 & 0.50 \\
Transportation, Storage, and Distributio & 11-3071 & 473 & Treatment & 0.85 & 0.71 \\
Purchasing Agents & 13-1023 & 407 & Treatment & 0.96 & 1.00 \\
Market Research Analysts and Marketing S & 13-1161 & 319 & Treatment & 1.88 & 1.00 \\
Librarians & 25-4021 & 314 & Treatment & 0.39 & 0.80 \\
Secondary School Teachers & 25-2030 & 313 & Treatment & 0.72 & 0.87 \\
Personal Financial Advisors & 13-2052 & 304 & Treatment & 1.57 & 1.00 \\
Computer and Information Systems Manager & 11-3021 & 301 & Treatment & 0.80 & 1.00 \\
Order Clerks & 43-4151 & 293 & Treatment & 0.17 & 1.00 \\
Special Education Teachers & 25-2050 & 252 & Treatment & 0.72 & 0.87 \\
Other Healthcare Practitioners & 29-9000 & 244 & Treatment & 0.64 & 0.67 \\
Other Personal Care Workers & 39-9099 & 240 & Treatment & 0.16 & 0.75 \\
Administrative Services Managers & 11-3011 & 237 & Treatment & 0.79 & 0.82 \\
Training and Development Specialists & 13-1151 & 234 & Treatment & 1.20 & 1.00 \\
Lawyers & 23-1011 & 234 & Treatment & 1.33 & 1.00 \\
Purchasing Managers & 11-3061 & 231 & Treatment & 0.99 & 1.00 \\
Electrical/Electronics Engineers & 17-2070 & 219 & Treatment & 1.08 & 0.78 \\
Logisticians & 13-1081 & 216 & Treatment & 1.03 & 1.00 \\
Broadcast/Sound Engineers & 27-4010 & 215 & Treatment & 0.48 & 0.83 \\
Speech-Language Pathologists & 29-1127 & 202 & Treatment & 1.46 & 1.00 \\
Physical Scientists, All Other & 19-2099 & 200 & Treatment & 1.38 & 1.00 \\
Human Resources Managers & 11-3121 & 193 & Treatment & 1.52 & 1.00 \\
Other Media and Comm. Workers & 27-3090 & 190 & Treatment & 0.51 & 0.56 \\
Probation Officers and Correctional Trea & 21-1092 & 183 & Treatment & 0.29 & 1.00 \\
Cost Estimators & 13-1051 & 165 & Treatment & 1.00 & 1.00 \\
Financial Analysts & 13-2051 & 163 & Treatment & 0.60 & 0.75 \\
Biological Scientists & 19-1020 & 163 & Treatment & 1.31 & 0.55 \\
Private Detectives and Investigators & 33-9021 & 161 & Treatment & 0.60 & 1.00 \\
Chief Executives & 11-1011 & 156 & Treatment & 1.96 & 1.00 \\
Editors & 27-3041 & 152 & Treatment & 0.59 & 1.00 \\
Misc. Mathematical Scientists & 15-2090 & 149 & Treatment & 1.23 & 1.00 \\
Operations Research Analysts & 15-2031 & 148 & Treatment & 1.39 & 1.00 \\
Photographers & 27-4021 & 147 & Treatment & 0.45 & 1.00 \\
Telemarketers & 41-9041 & 141 & Treatment & 0.54 & 1.00 \\
Securities, Commodities, and Financial S & 41-3031 & 137 & Treatment & 0.68 & 1.00 \\
News Analysts and Reporters & 27-3020 & 129 & Treatment & 0.51 & 0.56 \\
Artists and Related Workers & 27-1010 & 126 & Treatment & 0.23 & 0.82 \\
Architects & 17-1010 & 125 & Treatment & 0.96 & 0.85 \\
Environmental Scientists & 19-2040 & 120 & Treatment & 1.17 & 0.78 \\
Public Relations Specialists & 27-3031 & 118 & Treatment & 1.41 & 1.00 \\
Insurance Underwriters & 13-2053 & 116 & Treatment & 0.73 & 1.00 \\
Human Resources Assistants, Except Payro & 43-4161 & 114 & Treatment & 0.31 & 1.00 \\
Aerospace Engineers & 17-2011 & 103 & Treatment & 1.37 & 1.00 \\
Medical Scientists & 19-1040 & 102 & Treatment & 1.31 & 0.55 \\
Chemists and Materials Scientists & 19-2030 & 99 & Treatment & 1.17 & 0.78 \\
Surveyors and Cartographers & 17-1020 & 97 & Treatment & 0.96 & 0.85 \\
Compensation, Benefits, and Job Analysis & 13-1141 & 94 & Treatment & 0.52 & 1.00 \\
Misc. Social Scientists & 19-3090 & 93 & Treatment & 1.57 & 0.85 \\
Librarians & 25-4010 & 87 & Treatment & 0.39 & 0.80 \\
Misc. Media/Comm. Equipment Workers & 27-4030 & 86 & Treatment & 0.48 & 0.83 \\
Other Religious Workers & 21-2099 & 78 & Treatment & 0.40 & 1.00 \\
Psychologists & 19-3030 & 69 & Treatment & 1.57 & 0.85 \\
Financial Specialists, All Other & 13-2099 & 68 & Treatment & 1.09 & 1.00 \\
Directors, Religious Activities and Educ & 21-2021 & 65 & Treatment & 0.76 & 1.00 \\
Appraisers and Assessors of Real Estate & 13-2021 & 63 & Treatment & 0.60 & 0.75 \\
Conservation Scientists/Foresters & 19-1030 & 63 & Treatment & 1.31 & 0.55 \\
Cargo and Freight Agents & 43-5011 & 63 & Treatment & 0.23 & 1.00 \\
Technical Writers & 27-3042 & 59 & Treatment & 0.75 & 1.00 \\
Statistical Assistants & 43-9111 & 59 & Treatment & 0.86 & 1.00 \\
Announcers & 27-3010 & 55 & Treatment & 0.51 & 0.56 \\
Credit Analysts & 13-2041 & 54 & Treatment & 0.56 & 1.00 \\
Training and Development Managers & 11-3131 & 53 & Treatment & 1.55 & 1.00 \\
Architectural and Engineering Managers & 11-9041 & 46 & Treatment & 1.01 & 0.56 \\
Public Relations and Fundraising Manager & 11-2031 & 44 & Treatment & 0.93 & 1.00 \\
Recreational Therapists & 29-1125 & 40 & Treatment & 0.57 & 0.75 \\
Environmental Engineers & 17-2081 & 39 & Treatment & 0.96 & 1.00 \\
Computer Hardware Engineers & 17-2061 & 36 & Treatment & 1.04 & 1.00 \\
Agricultural and Food Scientists & 19-1010 & 34 & Treatment & 1.31 & 0.55 \\
Advertising and Promotions Managers & 11-2011 & 25 & Treatment & 0.33 & 1.00 \\
Judicial Law Clerks & 23-1012 & 22 & Treatment & 0.60 & 1.00 \\
Buyers and Purchasing Agents & 13-1021 & 21 & Treatment & 0.38 & 1.00 \\
Petroleum Engineers & 17-2171 & 21 & Treatment & 1.36 & 1.00 \\
Biomedical Engineers & 17-2031 & 19 & Treatment & 1.17 & 1.00 \\
Atmospheric and Space Scientists & 19-2021 & 18 & Treatment & 1.32 & 1.00 \\
Financial Examiners & 13-2061 & 15 & Treatment & 0.52 & 1.00 \\
Economists & 19-3011 & 15 & Treatment & 1.34 & 1.00 \\
Natural Sciences Managers & 11-9121 & 14 & Treatment & 0.88 & 1.00 \\
Compensation and Benefits Managers & 11-3111 & 13 & Treatment & 0.97 & 1.00 \\
Sales Engineers & 41-9031 & 13 & Treatment & 0.62 & 1.00 \\
Marine Engineers and Naval Architects & 17-2121 & 12 & Treatment & 0.59 & 0.52 \\
Astronomers and Physicists & 19-2010 & 7 & Treatment & 1.17 & 0.78 \\
Actuaries & 15-2011 & 5 & Treatment & 1.03 & 1.00 \\
Registered Nurses & 29-1141 & 9,223 & Control & 1.26 & 0.00 \\
Construction Laborers & 47-2061 & 3,885 & Control & 0.29 & 0.00 \\
Electricians & 47-2111 & 2,672 & Control & 0.76 & 0.00 \\
First-Line Supervisors of Production and & 51-1011 & 1,800 & Control & 0.16 & 0.00 \\
Maintenance and Repair Workers, General & 49-9071 & 1,780 & Control & 0.16 & 0.00 \\
Medical Assistants & 31-9092 & 1,759 & Control & 0.38 & 0.00 \\
Social Workers & 21-1020 & 1,587 & Control & 0.84 & 0.48 \\
Police and Sheriff's Patrol Officers & 33-3051 & 1,494 & Control & 0.61 & 0.00 \\
Counselors & 21-1010 & 1,435 & Control & 0.84 & 0.48 \\
Tellers & 43-3071 & 993 & Control & 0.18 & 0.00 \\
Chefs and Head Cooks & 35-1011 & 893 & Control & 0.86 & 0.00 \\
Correctional Officers and Jailers & 33-3010 & 825 & Control & 0.35 & 0.03 \\
Construction Managers & 11-9021 & 669 & Control & 0.61 & 0.00 \\
Social and Human Service Assistants & 21-1093 & 627 & Control & 0.12 & 0.00 \\
Education Administrators & 11-9030 & 591 & Control & 0.55 & 0.28 \\
Physical Therapists & 29-1123 & 542 & Control & 0.78 & 0.00 \\
Aircraft Mechanics and Service Technicia & 49-3011 & 530 & Control & 0.31 & 0.00 \\
First-Line Supervisors of Mechanics, Ins & 49-1011 & 484 & Control & 0.18 & 0.00 \\
First-Line Supervisors of Housekeeping a & 37-1011 & 477 & Control & 0.35 & 0.00 \\
Hairdressers, Hairstylists, and Cosmetol & 39-5012 & 472 & Control & 0.25 & 0.00 \\
Engineers, All Other & 17-2199 & 437 & Control & 0.78 & 0.47 \\
Phlebotomists & 31-9097 & 427 & Control & 0.19 & 0.00 \\
Water and Wastewater Treatment Plant and & 51-8031 & 424 & Control & 0.47 & 0.00 \\
Compliance Officers & 13-1041 & 370 & Control & 0.57 & 0.47 \\
Highway Maintenance Workers & 47-4051 & 358 & Control & 0.15 & 0.00 \\
Loan Interviewers and Clerks & 43-4131 & 340 & Control & 0.54 & 0.00 \\
Respiratory Therapists & 29-1126 & 334 & Control & 0.15 & 0.00 \\
Physical Therapist Assistants & 31-2020 & 328 & Control & 0.57 & 0.00 \\
Occupational Therapists & 29-1122 & 322 & Control & 0.83 & 0.00 \\
Misc. Physicians & 29-1129 & 298 & Control & 1.39 & 0.25 \\
Reservation and Transportation Ticket Ag & 43-4181 & 290 & Control & 0.18 & 0.00 \\
Nurse Practitioners & 29-1171 & 277 & Control & 1.26 & 0.00 \\
Mechanical Engineers & 17-2141 & 258 & Control & 0.86 & 0.27 \\
First-Line Supervisors of Landscaping, L & 37-1012 & 241 & Control & 0.14 & 0.00 \\
Industrial Production Managers & 11-3051 & 224 & Control & 0.57 & 0.00 \\
Dietitians and Nutritionists & 29-1031 & 211 & Control & 0.87 & 0.00 \\
Physicians and Surgeons & 29-1060 & 211 & Control & 0.76 & 0.00 \\
Detectives and Criminal Investigators & 33-3021 & 211 & Control & 0.84 & 0.21 \\
Eligibility Interviewers, Government Pro & 43-4061 & 206 & Control & 0.14 & 0.00 \\
Construction and Building Inspectors & 47-4011 & 196 & Control & 0.37 & 0.00 \\
First-Line Supervisors of Police and Det & 33-1012 & 195 & Control & 0.83 & 0.00 \\
Physician Assistants & 29-1071 & 194 & Control & 0.86 & 0.00 \\
Meeting, Convention, and Event Planners & 13-1121 & 191 & Control & 0.84 & 0.00 \\
Farmers, Ranchers, and Other Agricultura & 11-9013 & 185 & Control & 0.53 & 0.00 \\
Opticians, Dispensing & 29-2081 & 170 & Control & 0.94 & 0.00 \\
Industrial Engineers & 17-2110 & 161 & Control & 0.94 & 0.34 \\
Lodging Managers & 11-9081 & 156 & Control & 0.49 & 0.00 \\
Advertising Sales Agents & 41-3011 & 148 & Control & 0.55 & 0.00 \\
Tax Preparers & 13-2082 & 141 & Control & 0.32 & 0.00 \\
First-Line Supvrs, Protective Svc & 33-1099 & 130 & Control & 0.62 & 0.00 \\
Producers and Directors & 27-2012 & 125 & Control & 0.37 & 0.35 \\
Tax Examiners and Collectors, and Revenu & 13-2081 & 123 & Control & 0.24 & 0.00 \\
Occupational Therapy Assistants & 31-2010 & 114 & Control & 0.57 & 0.00 \\
First-Line Supervisors of Correctional O & 33-1011 & 105 & Control & 0.41 & 0.00 \\
First-Line Supervisors of Fire Fighting  & 33-1021 & 93 & Control & 0.61 & 0.00 \\
Biological Technicians & 19-4021 & 68 & Control & 0.27 & 0.00 \\
Fundraisers & 13-1131 & 60 & Control & 1.58 & 0.00 \\
Fire Inspectors and Prevention & 33-2020 & 60 & Control & 0.21 & 0.00 \\
Animal Trainers & 39-2011 & 54 & Control & 0.13 & 0.00 \\
Nurse Anesthetists & 29-1151 & 51 & Control & 1.19 & 0.00 \\
Radiation Therapists & 29-1124 & 50 & Control & 0.45 & 0.00 \\
Urban and Regional Planners & 19-3051 & 37 & Control & 1.68 & 0.00 \\
Veterinarians & 29-1131 & 37 & Control & 0.42 & 0.00 \\
Chemical Engineers & 17-2041 & 33 & Control & 1.19 & 0.00 \\
Optometrists & 29-1041 & 31 & Control & 1.32 & 0.00 \\
Parking Enforcement Workers & 33-3041 & 31 & Control & 0.16 & 0.00 \\
Gaming Managers & 11-9071 & 28 & Control & 0.27 & 0.00 \\
Morticians, Undertakers, and Funeral Dir & 39-4031 & 28 & Control & 0.38 & 0.00 \\
Audiologists & 29-1181 & 26 & Control & 1.25 & 0.00 \\
Materials Engineers & 17-2131 & 25 & Control & 1.12 & 0.00 \\
Health Diagnosing and Treating Practitio & 29-1199 & 21 & Control & 0.95 & 0.14 \\
Emergency Management Directors & 11-9161 & 17 & Control & 1.04 & 0.00 \\
Dentists & 29-1020 & 15 & Control & 0.76 & 0.00 \\
Chiropractors & 29-1011 & 12 & Control & 0.49 & 0.00 \\
Electronic Equipment Installers and Repa & 49-2096 & 11 & Control & 0.31 & 0.00 \\
\end{longtable}

\begin{minipage}{0.95\textwidth}
\small\textit{Notes:} Occupations sorted by sample size within each group. Abstract is the standardized O*NET abstract-task index. TW Score is the Dingel-Neiman telework feasibility score. Entries labeled ``SOC XX-XXXX'' represent aggregated occupation categories in the IPUMS CPS coding that lack a direct match to detailed O*NET occupation titles.
\end{minipage}


\section{Additional Figures}

\begin{figure}[H]
\centering
\caption{Event Study by Metropolitan Status}
\label{fig:metro_es}
\includegraphics[width=0.85\textwidth]{fig_A1_metro_nonmetro_es.pdf}

\bigskip
\begin{minipage}{0.95\textwidth}
\small\textit{Notes:} The data are from CPS-ORG, 2017Q1--2023Q1. This figure shows separate event-study estimates for metropolitan (blue) and non-metropolitan (orange) subsamples. Both specifications include occupation, year-quarter, education, race, industry, and state fixed effects. Standard errors clustered at the occupation level. Reference period: 2019Q4.
\end{minipage}
\end{figure}


% ==============================================================================
\end{document}
% ==============================================================================
